\documentclass[a4paper,11pt]{article}
\usepackage{indentfirst}
\setlength\parindent{24pt}
\usepackage[T1]{fontenc}
\usepackage[polish]{babel}
\usepackage[utf8]{inputenc}
\usepackage{lmodern}
\selectlanguage{polish}
\usepackage[top=2cm, bottom=2cm, left=1cm, right=1cm]{geometry}
\usepackage{lastpage}
\usepackage{fancyhdr}
\pagestyle{fancy}
\makeatletter
\newcommand{\linia}{\rule{\linewidth}{0.4mm}}
\renewcommand{\maketitle}{\begin{titlepage}
    \vspace*{2cm}
    \begin{center}\LARGE
    Politechnika Warszawska\\
    Wydział Elektryczny\\
    \end{center}
    \vspace{5cm}
    \noindent\linia
    \begin{center}
      \LARGE \textsc{\@title}
         \end{center}
     \linia
    \vspace{0.5cm}
    \begin{flushright}
    \begin{minipage}{5cm}
    \textit{Autor:}\\
    \normalsize \textsc{\@author} \par
    \end{minipage}
    \vspace{5cm}
     \end{flushright}
    \vspace*{\stretch{6}}
    \begin{center}
    \@date
    \end{center}
  \end{titlepage}
}
\makeatother
\author{Grzegorz Kopyt\\Arkadiusz Michalak}
\title{Specyfikacja Implementacyjna}
\usepackage{graphicx}
\fancyhf{}
\rfoot{\thepage{}/\pageref{LastPage}}
\begin{document}

\maketitle

\tableofcontents
\vspace{1cm}
\noindent\linia

\section{Wstęp teoretyczny}

\noindent\linia
\section{Diagram klas}

\noindent\linia
\section{Opis algorytmu}


\noindent\linia
\section{Opis ważniejszych metod}
\begin{itemize}
\item 
\end{itemize}

\noindent\linia

\section{Testy}
\begin{itemize}
\item 
\end{itemize}

\noindent\linia
\section{Informacje o sprzęcie i oprogramowaniu}
Program będzie pisany w języku C\# na platformie .NET z wykorzystaniem WPF oraz środowiska Visual~Studio (15.9.4).

Zostanie przetestowany na komputerach:
\begin{enumerate}
\item
\end{enumerate}
\noindent\linia

\end{document}




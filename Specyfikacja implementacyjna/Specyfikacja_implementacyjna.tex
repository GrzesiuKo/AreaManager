\documentclass[a4paper,11pt]{article}
\usepackage{indentfirst}
\setlength\parindent{24pt}
\usepackage[T1]{fontenc}
\usepackage[polish]{babel}
\usepackage[utf8]{inputenc}
\usepackage{lmodern}
\selectlanguage{polish}
\usepackage[top=2cm, bottom=2cm, left=1cm, right=1cm]{geometry}
\usepackage{lastpage}
\usepackage{fancyhdr}
\pagestyle{fancy}
\makeatletter
\newcommand{\linia}{\rule{\linewidth}{0.4mm}}
\renewcommand{\maketitle}{\begin{titlepage}
    \vspace*{2cm}
    \begin{center}\LARGE
    Politechnika Warszawska\\
    Wydział Elektryczny\\
    \end{center}
    \vspace{5cm}
    \noindent\linia
    \begin{center}
      \LARGE \textsc{\@title}
         \end{center}
     \linia
    \vspace{0.5cm}
    \begin{flushright}
    \begin{minipage}{5cm}
    \textit{Autor:}\\
    \normalsize \textsc{\@author} \par
    \end{minipage}
    \vspace{5cm}
     \end{flushright}
    \vspace*{\stretch{6}}
    \begin{center}
    \@date
    \end{center}
  \end{titlepage}
}
\makeatother
\author{Grzegorz Kopyt\\Arkadiusz Michalak}
\title{Specyfikacja Implementacyjna}
\usepackage{graphicx}
\fancyhf{}
\rfoot{\thepage{}/\pageref{LastPage}}
\begin{document}

\maketitle

\tableofcontents
\vspace{1cm}
\noindent\linia

\section{Wstęp teoretyczny}

\noindent\linia
\section{Diagram klas}

\noindent\linia
\section{Opis algorytmu}
\subsection{Sprawdzanie wypukłości konturu}
Wszystkie punkty konturu podane przez użytkownika przechowywane będą w klasie \textit{Contour}.

Na podstawie algorytmu Jarvisa zostaną one podzielone na te, które wejdą w skład konturu oraz na te, które zostaną zignorowane. Punkty będą ignorowane, jeśli podany kontur nie będzie wypukły. Wtedy algorytm stworzy wypukły kontur, na podstawie podanych punktów, ignorując te, które uzna za burzące wypukły kształt figury. Punkty wchodzące w skład konturu zostaną zachowane w kolejności łączenia.


\noindent\linia
\section{Opis ważniejszych metod}
\begin{itemize}
\item 
\end{itemize}

\noindent\linia

\section{Testy}
\begin{itemize}
\item 
\end{itemize}

\noindent\linia
\section{Informacje o sprzęcie i oprogramowaniu}
Program będzie pisany w języku C\# na platformie .NET z wykorzystaniem WPF oraz środowiska Visual~Studio (15.9.4).

Zostanie przetestowany na komputerach:
\begin{enumerate}
\item Lenovo G510 o procesorze Intel Core i5 2.5GHz, pamięci RAM 6GB, karcie graficznej AMD Radeon
HD 8570M i systemie operacyjnym Windows 10,
\item Asus X7500J o procesorze Intel Core i7 2.4GHz, pamięci RAM 8GB, karcie graficznej NVIDIA GeForce GT 740M i systemie operacyjnym Windows 10. 
\end{enumerate} 
\noindent\linia

\end{document}




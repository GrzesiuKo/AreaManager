\documentclass[a4paper,11pt]{article}
\usepackage{indentfirst}
\usepackage[T1]{fontenc}
\usepackage[polish]{babel}
\usepackage[utf8]{inputenc}
\usepackage{lmodern}
\selectlanguage{polish}
\usepackage[top=2cm, bottom=2cm, left=1cm, right=1cm]{geometry}
\usepackage{lastpage}
\usepackage{fancyhdr}
\pagestyle{fancy}
\setlength\parindent{24pt}
\makeatletter
\newcommand{\linia}{\rule{\linewidth}{0.4mm}}
\renewcommand{\maketitle}{\begin{titlepage}
    \vspace*{2cm}
    \begin{center}\LARGE
    Politechnika Warszawska\\
    Wydział Elektryczny\\
    \end{center}
    \vspace{5cm}
    \noindent\linia
    \begin{center}
      \LARGE \textsc{\@title}
         \end{center}
     \linia
    \vspace{0.5cm}
    \begin{flushright}
    \begin{minipage}{5cm}
    \textit{Autor:}\\
    \normalsize \textsc{\@author} \par
    \end{minipage}
    \vspace{5cm}
     \end{flushright}
    \vspace*{\stretch{6}}
    \begin{center}
    \@date
    \end{center}
  \end{titlepage}
}
\makeatother
\author{Grzegorz Kopyt \\ Arkadiusz Michalak}
\title{Specyfikacja Funkcjonalna\\Projekt Zespołowy 2018/2019}
\usepackage{graphicx}

\fancyhf{}
\rfoot{\thepage{}/\pageref{LastPage}}

\begin{document}
\maketitle

\tableofcontents
\vspace{1cm}
\noindent\linia
\section{Wstęp teoretyczny}
Dokument ten dotyczy programu realizowanego w ramach ,,Projektu Zespołowego 2018/2019".

Głównym zadaniem jest analiza i podział terenu na optymalne części. Program na podstawie podanego konturu terenu oraz punktów kluczowych, znajdujących się na tym terenie, powinien podzielić go na optymalne części. Oznacza to, że każdy z powstałych obszarów powinien zawierać jeden punkt kluczowy, a granice powinny obejmować każde miejsce, z którego bliżej jest do danego punktu kluczowego niż do jakiegokolwiek innego z~punktów kluczowych.

Dodatkowo na całą mapę zostaną naniesione różne typy obiektów (między innymi domy z mieszkańcami), a~program powinien przygotować statystykę ilości obiektów oraz mieszkańców na danej części terenu.

Pozostałe funkcje programu zostały opisane w sekcji ,,Wymagania funkcjonalne".

\noindent\linia
\section{Wymagania funkcjonalne}

Program powinien spełniać podane wymagania funkcjonalne:
\begin{itemize}
\item podanie konturu terenu wraz z rozmieszczeniem punktów kluczowych,
\item podzielenie terenu na optymalne obszary,
\item narysowanie granic optymalnych obszarów,
\item naniesienie na wczytany teren obiektów,
\item podawanie obiektów i definiowanie ich typów,
\item dodawanie i usuwanie elementów konturu terenu,
\item dodawanie i usuwanie punktów kluczowych,
\item nakładanie grafiki pod wyznaczone kontury,
\item wyświetlanie listy obiektów należących do danego obszaru,
\item wyświetlanie zbiorcze listy obiektów należących do danego obszaru,
\item wyświetlanie liczby mieszkańców danego obszaru.
\end{itemize}

\noindent\linia
\section{Obsługa}
\begin{itemize}
\item
\end{itemize}

\noindent\linia
\section{Komunikaty o błędach}
\begin{itemize}
\item
\end{itemize}

\noindent\linia
\section{Testy akceptacyjne}
\begin{itemize}
\item
\end{itemize}
\noindent\linia

\end{document}



